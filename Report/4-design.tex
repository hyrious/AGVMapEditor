% !TeX root = ./report.tex

本章是软件开发的概要设计和详细设计,主要目标是以需求分析为依据,记录系统整体与细节上的具体实现上的考虑。本章将作为产品实现、特性要求和控制的依据。

\section{概要设计}

\subsection{服务模型}

即数据、功能与用户之间的关系,本地图编辑器的服务模型分为三层:用户、业务逻辑和数据服务。用户层提供交互界面并调用业务 API,业务逻辑层处理数据并向用户提供相关功能的接口,数据服务由使用的数据库实现。

\begin{figure}[H]
  \centering
  \includegraphics[width=0.8\textwidth]{assets/serv.pdf}
  \caption{服务模型}
  \label{fig:serv}
\end{figure}

\subsection{数据流图}

\begin{figure}[H]
  \centering
  \includegraphics[width=0.8\textwidth]{assets/dataflow.pdf}
  \caption{数据流图}
  \label{fig:dataflow}
\end{figure}

\subsection{流程图}

\begin{figure}[H]
  \centering
  \includegraphics[width=0.4\textwidth]{assets/flow.pdf}
  \caption{流程图}
  \label{fig:flow}
\end{figure}

\section{详细设计}

\subsection{系统数据库设计}

本系统追求简洁方便,总共涉及一张表,即存储了所有坐标点和它上面的属性,包括各个方向上的小车姿态等,详细设计如下:

\subsubsection{数据库 E-R 图}

\begin{figure}[H]
  \centering
  \includegraphics[width=0.6\textwidth]{assets/er.pdf}
  \caption{数据库 E-R 图}
  \label{fig:er}
\end{figure}

\subsubsection{数据表设计}

\paragraph{coords}

该表用来存储所有坐标点的所有数据,设计如表:

\begin{longtabu}to \textwidth{rllX}
  \toprule
  列名 & 数据类型 & 能否为空 & 描述 \\
  \midrule
  \endhead
  \bottomrule
  \endfoot
  id & \texttt{integer} & 否 & 创建时生成的唯一 \texttt{ID},默认为递增 \\
  x & \texttt{integer} & 否 & (相对于画布的)X 坐标 \\
  y & \texttt{integer} & 否 & (相对于画布的)Y 坐标 \\
  position\_barcode & \texttt{integer} & 否 & 坐标点二维码 \\
  up\_point\_id & \texttt{integer} & 是 & 指向上方的点的 \texttt{ID},为空或 0 代表没有连线 \\
  up\_check1\_mode & \texttt{integer} & 是 & 向上方运动时检测 1 的模式,取 1-7 七种 \\
  up\_check2\_mode & \texttt{integer} & 是 & 向上方运动时检测 2 的模式,取 1-7 七种 \\
  up\_pose & \texttt{integer} & 是 & 向上方运动时的姿态,取 0、1 分别代表前进后退 \\
  down\_point\_id & \texttt{integer} & 是 & 指向下方的点的 \texttt{ID},为空或 0 代表没有连线 \\
  down\_check1\_mode & \texttt{integer} & 是 & 向下方运动时检测 1 的模式,取 1-7 七种 \\
  down\_check2\_mode & \texttt{integer} & 是 & 向下方运动时检测 2 的模式,取 1-7 七种 \\
  down\_pose & \texttt{integer} & 是 & 向下方运动时的姿态,取 0、1 分别代表前进后退 \\
  left\_point\_id & \texttt{integer} & 是 & 指向左方的点的 \texttt{ID},为空或 0 代表没有连线 \\
  left\_check1\_mode & \texttt{integer} & 是 & 向左方运动时检测 1 的模式,取 1-7 七种 \\
  left\_check2\_mode & \texttt{integer} & 是 & 向左方运动时检测 2 的模式,取 1-7 七种 \\
  left\_pose & \texttt{integer} & 是 & 向左方运动时的姿态,取 0、1 分别代表前进后退 \\
  right\_point\_id & \texttt{integer} & 是 & 指向右方的点的 \texttt{ID},为空或 0 代表没有连线 \\
  right\_check1\_mode & \texttt{integer} & 是 & 向右方运动时检测 1 的模式,取 1-7 七种 \\
  right\_check2\_mode & \texttt{integer} & 是 & 向右方运动时检测 2 的模式,取 1-7 七种 \\
  right\_pose & \texttt{integer} & 是 & 向右方运动时的姿态,取 0、1 分别代表前进后退
\end{longtabu}

\section{本章总结}

本章展现了设计上的一些阶段性成果,基本概括了软件的结构和实现形式。下一章将通过一些真实的代码来介绍具体的实现。
