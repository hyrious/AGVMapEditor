% !TeX root = ./report.tex

本章是软件开发的概要设计和详细设计,主要目标是以需求分析为依据,记录系统整体与细节上的具体实现上的考虑。本章将作为实现、特性要求和控制的依据。

\section{概要设计}

\subsection{服务模型}

即数据、功能与用户之间的关系,本地图编辑器的服务模型分为三层:用户、业务逻辑和数据服务。用户层提供交互界面并调用业务 API,业务逻辑层处理数据并向用户提供相关功能的接口,数据服务由使用的数据库实现。

\begin{figure}[H]
  \centering
  \includegraphics[width=0.6\textwidth]{assets/serv.pdf}
  \caption{服务模型}
  \label{fig:serv}
\end{figure}

写过一些程序的读者可能会发现,上图就是我们熟知的 MVC 模式。作为一个本地桌面程序,本软件的``服务''在于用户操作和数据库文件读写之间的关系,因此直接采用一种简化的 MVC 模型即可。

\subsection{数据流图}

借用 Martin Fowler 的 GUI 架构一文\cite{url:uiArchs}所述,通常的 GUI 程序存有三份数据——数据库的、软件内存的、界面上刚刚修改过还未同步到内存中的。本软件的情形类似,下图中展示了本软件各种数据的流向概况。

\begin{figure}[H]
  \centering
  \includegraphics[width=0.6\textwidth]{assets/dataflow.pdf}
  \caption{数据流图}
  \label{fig:dataflow}
\end{figure}

\subsection{流程图}

整个软件的流程图大致如下:

\begin{figure}[H]
  \centering
  \includegraphics[width=0.4\textwidth]{assets/flow.pdf}
  \caption{流程图}
  \label{fig:flow}
\end{figure}

\section{详细设计}

\subsection{系统数据库设计}

数据/原型设计是任何软件程序开发的前提和基础,本软件主要涉及到对二维坐标点及其相关属性的存储,相对其他中型平台来说较为简单。数据库选择上,本软件将采用 LiteDB 作为轻量级存储,LiteDB 因为和 .NET 平台对接方便,接口易用,因此被我选来作为本软件的数据存储功能。因为没有作为服务器运行的必要,本软件将直接在客户计算机上离线运行。

本系统追求实现简便,总共涉及一张表,即存储了所有坐标点和它上面的属性,包括各个方向上的小车姿态等,详细设计如下:

\subsubsection{数据库 E-R 图}

\begin{figure}[H]
  \centering
  \includegraphics[width=0.6\textwidth]{assets/er.pdf}
  \caption{数据库 E-R 图}
  \label{fig:er}
\end{figure}

\subsubsection{数据表设计}

\paragraph{coords}

该表用来存储所有坐标点的所有数据,设计如表:

\begin{longtabu}to \textwidth{rllX}
  \toprule
  列名 & 数据类型 & 能否为空 & 描述 \\
  \midrule
  \endhead
  \bottomrule
  \endfoot
  id & \texttt{integer} & 否 & 创建时生成的唯一 \texttt{ID},默认为递增 \\
  x & \texttt{integer} & 否 & (相对于画布的)X 坐标 \\
  y & \texttt{integer} & 否 & (相对于画布的)Y 坐标 \\
  position\_barcode & \texttt{integer} & 否 & 坐标点二维码 \\
  up\_point\_id & \texttt{integer} & 是 & 指向上方的点的 \texttt{ID},为空或 0 代表没有连线 \\
  up\_check1\_mode & \texttt{integer} & 是 & 向上方运动时检测 1 的模式,共七种 \\
  up\_check2\_mode & \texttt{integer} & 是 & 向上方运动时检测 2 的模式,共七种 \\
  up\_pose & \texttt{integer} & 是 & 向上方运动时的姿态,取 0、1 分别代表前进后退 \\
  down\_point\_id & \texttt{integer} & 是 & 指向下方的点的 \texttt{ID},为空或 0 代表没有连线 \\
  down\_check1\_mode & \texttt{integer} & 是 & 向下方运动时检测 1 的模式,共七种 \\
  down\_check2\_mode & \texttt{integer} & 是 & 向下方运动时检测 2 的模式,共七种 \\
  down\_pose & \texttt{integer} & 是 & 向下方运动时的姿态,取 0、1 分别代表前进后退 \\
  left\_point\_id & \texttt{integer} & 是 & 指向左方的点的 \texttt{ID},为空或 0 代表没有连线 \\
  left\_check1\_mode & \texttt{integer} & 是 & 向左方运动时检测 1 的模式,共七种 \\
  left\_check2\_mode & \texttt{integer} & 是 & 向左方运动时检测 2 的模式,共七种 \\
  left\_pose & \texttt{integer} & 是 & 向左方运动时的姿态,取 0、1 分别代表前进后退 \\
  right\_point\_id & \texttt{integer} & 是 & 指向右方的点的 \texttt{ID},为空或 0 代表没有连线 \\
  right\_check1\_mode & \texttt{integer} & 是 & 向右方运动时检测 1 的模式,共七种 \\
  right\_check2\_mode & \texttt{integer} & 是 & 向右方运动时检测 2 的模式,共七种 \\
  right\_pose & \texttt{integer} & 是 & 向右方运动时的姿态,取 0、1 分别代表前进后退
\end{longtabu}

\subsection{界面设计}

界面是大部分软件的用户能够接触到的唯一部分,因此占了用户评价的绝大多数比重。本软件作为本地桌面 GUI 程序,将使用 WPF 框架对软件界面进行设计,力图做到使用方便、操作自然、图形精美。

本软件的界面分为以下两个部分:

\subsubsection{可交互画布}

指的是用户通过鼠标键盘等输入设备直接在本软件窗体上描绘坐标点及连线的区域。为使后期开发方便,辅助理清用户操作逻辑,做出如下设计:该区域可以捕捉用户的鼠标和键盘输入,并作出相应反馈。针对不同的输入,有

\begin{enumerate}
  \item 双击空白区域则在该位置创建一个新的坐标点
  \item 双击坐标点则打开其属性设置窗口
  \item 单击某坐标点则选中,单击空白区域则取消选中
  \item 右击空白区域按住则拖动画布
  \item 右击某坐标点按住则创建连线,并自动确定方向
  \item 按住 Ctrl 则取消自动对齐
  \item 按 Delete 则删除当前选中点
\end{enumerate}

针对界面上显示的内容,有

\begin{enumerate}
  \item 显示每个坐标点及其属性
  \item 显示每条连线及其方向、长度
\end{enumerate}

\subsubsection{菜单栏}

以上提到的每一个功能,都在菜单栏中留有选项,并且增加如下选项:

\begin{enumerate}
  \item 打开/保存/另存为到数据文件
  \item 导出 SQL 语句
\end{enumerate}

\section{本章总结}

本章展现了设计上的一些阶段性成果,基本概括了软件的结构和实现形式。下一章将通过一些真实的软件截图和代码片段来介绍具体的实现。
