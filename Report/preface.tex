% !TeX root = ./report.tex

工业 4.0 以来,机器人以其高效、准确、便宜在各行各业获得了广泛的应用。有一类机器人的需求比较普遍,就是定位和移动。说白了就是要解决``我在哪'' ``我要去哪'' ``怎么去''的问题。传统的定位方式一般是磁条、磁钉之类,路径基本是固定的。而现在一般会用到类似 GPS 的高级的定位方式,然后在预设的地图中按设定的路线移动。当然也有更高级的机器人,可以通过对环境建模、然后通过视觉匹配等技术得到位置和移动目标。本软件是为了开发一个界面友好的地图编辑软件,作为 AGV 寻路和调度的基础设施。

开发细节上,本地图编辑软件使用了 WPF(C\#) 来开发 Windows 平台上的原生桌面程序。虽然现在开发图形程序更多的会提到网页前端,而且 Electron、Flutter 等利用网页技术开发原生桌面程序的技术也得到了大量关注,但是课题要求使用 C\#。为了防止 HTML/CSS/JS 代码的比重过高,我决定使用 Windows 平台的原生 GUI 开发框架 WPF 来制作这个软件。依赖于 WPF 的优秀组件库,基本上这个软件的需求可以只用 WPF 自带的 Canvas 等组件来完成,而且可以做到美观。然后为了持久化保存地图数据这里使用了一个轻量的 NoSQL 库 LiteDB,这样可以让工作重心放在处理和显示地图数据上而不是折腾数据库上。开发环境使用 Visual Studio,版本管理工具使用 Git 并开源到 GitHub\cite{url:ghrepo}。

本文将对整个软件的开发过程,包括各种前期准备工作,来进行描述。主要分为六个部分来介绍:

\begin{enumerate}
  \item 绪论,包括软件开发的背景和研究现状。
  \item 需求分析和可行性分析,将从多个角度来阐述需求和可行性。
  \item 相关技术,系统化地描述、整理了相关信息。
  \item 系统设计,包括数据库、GUI 逻辑设计等。
  \item 系统实现,对每个功能的实现细节的分析。
  \item 总结,开发过程中的收获和感悟及对 GUI、AVG 等技术未来的展望。
\end{enumerate}
