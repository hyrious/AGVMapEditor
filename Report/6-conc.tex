% !TeX root = ./report.tex

\section{总结}

经过三个月的学习、开发,终于完成了计划中的每一项任务。虽然原本我的个人兴趣在前端——网页开发技术等上面,但是为了本次毕业设计也是下力气学了一下 C\# 和 WPF 开发。WPF 这个技术从 Vista 就开始有了,但是它的推广和发展并不顺利。至少中文学习资料只有一两本书,用的人除了微软自己也不算多。我想这主要是因为其晦涩的 XAML 编程和无法解决的性能瓶颈导致的。C\# 和 WPF 都是设计思想十分超前的东西,但是超前不意味着流行。Lisp 中蕴含的闭包等程序思想深刻影响了后世出现的种种编程语言,但是由于``新语言问题''\cite{url:yinwang},Lisp 最终还是淡出了大众视野,目前还在大量使用类 Lisp 的地方只有作为 Emacs 的插件语言以及用 Racket 创造新工具而已。C\# 则是引入了太多语言特性,当然它们确实解决了 Java 的许多痛点,但是也使得 C\# 沦为一个``脚本语言'',你现在还可以在 Unity 等地方看到 C\# 活跃的身影。如果本软件还有二次开发的需求的话,我个人建议是采用 Web 应用或者类似方式来实现用户界面,一来现在前端人口众多,开源库丰富;二来就本次毕业设计而言二次开发的成本更低,相同的需求可以用更直观简单的代码来实现,现在的代码其实有些高耦合了。

本次毕业设计中,我也用到了一些代码生成的技术,来帮助我减少因为手误产生的 bug。当然这个技术是没有出现在最终代码里的,出现的都是生成后的代码。目前 IT 行业发展不算好,``业界寒冬''我已经听了许多次,我也见到不少或主动或被动辞职的程序员,过量的加班压力是促成这一事实的重要因素之一。我认为在行业危机面前,提升自己的熟练度和``有效代码生成速度''是有效避免加班的关键。

\section{展望}

AGV 虽然目前还属于高新技术,但是已然不新鲜了。更多的人和公司投入到这个产业中来分一杯羹,是看好了其高效、廉价、安全的生产力。未来是机器人的时代,人力成本正在日益上升,从最近火热的 996.icu\cite{url:996icu} 事件可见一斑。相信随着人口老龄化等社会进程,机器人代替人类工作的势头必一发不可收拾。只要谁能开发出更廉价高效的工业器械就能尝试在这次工业革命中占据领跑地位。当然本次毕业设计并不涉及新工业产品的研发,只是做了个小前端的工作而已。

目前软件的功能相对简单,只是完成了一些基本的功能。换句话说,本软件还有很大扩展进步空间,主要包括:

\begin{enumerate}
  \item 优化 WPF 性能,这一步对于目前的我来说几乎不可能有任何进展。如果一定要进行优化的话建议换一个现代框架如 Electron 等。
  \item 画布缩放,因为坐标算起来太麻烦了所以这一功能没能实现。
  \item 添加其他图形和注释,需要增加若干数据表项和相关描绘逻辑。
  \item 开发一个数据库后端,支持多人协作编辑。这个特性需要把描绘的逻辑一部分搬到后端实现,所以单纯的 WPF 不够。另外涉及到协作逻辑,可能实现难度比上面三个都要高。
  \item 美化界面样式,本人审美不行,美术能力为 0,因此做出来的东西也看着十分缺乏设计。
\end{enumerate}

希望自己能够不断积累软件程序开发的相关知识,然后用学成的知识开发出优秀的软件以方便一部分人的日常和工作需求。
