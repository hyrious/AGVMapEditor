% !TeX root = ./report.tex

本章是就开发基于 WPF 的 AGV 地图编辑器的可行性及需求分析。将对使用到的技术进行描述和分析,对开发过程中遇到的技术问题进行阐述,以及对将来可能遇到的需求的思考。

\section{可行性分析}

\subsection{技术可行性}

\begin{enumerate}
  \item 开发技术\\
        现在有许多开发 GUI 的技术,如 Qt、WinForm、Electron 等。但本项目采用 WPF 这种不被大众普遍看好的开发框架。其优点是有一定的自定义组件的能力,以及其中一个组件恰好符合本次毕业设计需要的特性。维护成本低、部署容易、不跨平台。
  \item C\# 语言的开发形式\\
        整个项目仅包含 XAML 和 C\# 两种语言(尽管前者并非编程语言),这样便于降低代码学习的总量,使得开发过程连贯可控,减少代码的重复率,加快开发速度。其次,在框架选择中,我对 WinForm、WPF、UWP 等使用 C\# 的平台进行了比较,考虑到项目的规模以及 UWP 的部署方式并不简单,最终采用了纯 WPF 框架来进行代码的组织。
  \item 项目应用平台\\
        仅支持 Windows 平台,而且用户机器上需要安装 .NET 运行库(新版 Windows 上自带)。优点在于使用了``托管代码''的软件本体只有不超过 1M 的大小,美中不足的是启动时间较长、内存占用较多。
\end{enumerate}

\subsection{组织可行性}

GUI 开发有众多案例参考,对于本项目的规模,结合案例分析,在项目的计划时间内,项目可以完全投入测试环境运作。虽然要想和其他产品竞争还有待时日,不过可以保证项目顺利进行。整体的开发计划如下:

\begin{enumerate}
  \item 准备工作,了解大量关于 C\#、WPF 的知识。收集相关资料完成学习,包括 WPF 的主要控件和布局方式、XAML 的语法和特性、C\# 的语法和特性、相关库的使用如连接数据库等,大约需要两个月的时间。
  \item 进入开发状态,把本项目的每个功能模块分配一周时间来进行前后端(对于 GUI 软件而言,前端就是布局、后端就是用户操作下对应的功能)开发。由于熟练度不高,可能需要很长时间来完成。初步估计安排三周。
  \item 收尾,整理项目清单,完成毕业论文。由于本人写文捉襟见肘,这一步就留剩下所有时间来完成。
\end{enumerate}

\subsection{时间可行性}

本次毕业设计的开发时间比较充足,基本上大四下学期的 3 个月都可以用来学习开发。由于之前没有接触过 WPF 的开发,所以安排了较长时间进行学习,以期在开发阶段能够少出 Bug。总体来说时间是足够的,因此本毕业设计在时间上是可行的,具体时间安排可以参考上一节``组织可行性''。

\subsection{经济可行性}

开发本毕设由于只用了免费的开发工具和开源的库,所以消耗的资金基本为 0,不过耗费了许多个人时间,时间也是金钱嘛。对于客户端来讲,只要是个 Windows 系统的电脑就可以使用本软件。对于服务端来讲,本软件使用了本地文件数据库 LiteDB,因此没有维护数据服务器的必要,这个需求可以等有了再去购买和部署。由于本软件并没有交付 WinXP 系统使用者的需求和必要,因此本毕设在经济上也是可行的。

\subsection{结论}

综合以上四点,本软件无论从技术、组织、时间还是经济上分析都是可行的。只要按部就班、按图索骥,准备充分、认真实现,不出意外的话就能如期完成。

\section{需求分析}

本节将对 WPF 实现的 AGV 地图编辑器的四个基本需求进行描述,分别为功能需求、性能需求、可靠可用性需求和将来可能的需求。

\subsection{功能需求分析}

作为一个地图编辑器,首先需要能够读取、修改和保存地图数据。读取和保存自不必多言,对于修改数据而言有一些必要的功能,如能够修改点的位置、点的坐标``二维码''、点与点之间连线的状态和相关属性。通过上述分析,本地图编辑器的功能需求按显示-逻辑而言分为以下几条:

\begin{itemize}
  \item 显示
  \begin{enumerate}
    \item 显示每个坐标点到对应的位置
    \item 显示每个有向连线到对应的位置
    \item 在图上尽可能地显示每个点上的属性
  \end{enumerate}
  \item 逻辑
  \begin{enumerate}
    \item 用户可以新建一个坐标点
    \item 用户可以将上一次选中的点删掉,同时删掉对应的连线
    \item 用户可以拖动画布
    \item 用户可以拖动坐标点
    \item 用户可以编辑每个点上的属性信息
    \item 用户按保存后将把整个图数据同步到数据库
  \end{enumerate}
\end{itemize}

\subsection{性能需求分析}

本软件并没有多人协作或者并发访问数据库的需求,因此不需要考虑这方面的性能问题。对于 GUI 程序而言,性能更多的体现在冷启动速度和操作流畅度上。对于 .NET 程序,其启动速度视首屏控件数量而定。这一点上要做到尽量少的首屏控件,让大部分控件做到在 Loaded 事件中延迟加载。至于操作流畅度,对于 WPF 而言基本没有较好的通用解决办法。因为其控件都是被封装好的,要想针对某个控件进行优化是一件 Hack 的事情,最常见的比如你想加个控件阴影,写了个 \texttt{<DropShadowEffect>}。嗯,效果不错,不过为什么软件突然就变卡了?原来是因为这个阴影是通过 CPU 进行每帧的运算描绘上去的,常见的家用 CPU 一般很难在这个使用频率下不卡,因此最终我把这部分代码从源码中移除了。对 WPF 原理研究深入的人可以把它改成 GPU 或者带缓存的描绘策略,但是我目前的水平搞不定。这个问题的重点是你抄了一堆自己也看不懂的 Hack,最终也不一定能达到你想要的需求,换句话说,这些工作应该交给微软做,不是么(然后他就会建议你使用 WebView)。另一个影响流畅度的地方是进行数据库读写的时候往往会卡住进程,这里就要使用 C\# 提供的 async 语法来异步地进行操作。

\subsection{可靠性和可用性}

本软件并非线上系统,也没有用户权限管理等需求。因此不存在泄露用户信息等安全问题。本软件使用的是本地文件数据库,因此不存在数据服务器并发或者泄露的问题。当然还是有用户自己的电脑损坏或重装等情况导致数据丢失,这里也提供了导出到其他数据库的备份策略。本次开发由于个人水平有限,只在一些微小的方面做了贡献。而软件的可用性指的是软件能够满足开发最初预期的需求,功能是否齐全,操作是否简便。在界面设计上,考虑通用软件的使用习惯,尽量提高本软件的可用性。

\subsection{将来可能增加的需求}

由于目前个人水平有限,本次毕业设计开发的软件还不完善,只是最小限度地支持了预期的功能需求。另外还有一些看似简单的需求不在预期需求里面,这主要还是由于对 WPF 的熟练度不够导致的,但我也不想花大力气去填这些坑了,如果以后还有机会碰 WPF 的话再去思考如何实现吧。这些需求有:

\begin{itemize}
  \item 画布缩放
  \item 添加其他图形和注释
  \item 开发一个数据库后端,支持多人协作编辑
  \item 美化界面样式
\end{itemize}

\section{本章总结}

本章主要对开发 AGV 地图编辑器的可行性和需求分析做了详细的描述,总结了主要的功能、存在的问题和对未来开发的需求。从上述分析得知,目前本软件在设计上还是存在一些漏洞,一些功能的实现和预期目标还有细微的差别。性能问题上,由于本人还缺乏一定的开发经验,以及 WPF 的资料并不丰富,因此还是无法将其优化到完美状态。组织分析中,由于本次毕业设计是个人独立开发项目,所以没有涉及到团队合作的开发。开发软件当然是可以一个人独立完成的,但是只有团队才能照顾到一个软件日后的生态,例如推广和迭代升级等。在接口问题上,虽然本软件没有采用广为大众熟知的数据库软件后端如 SQL Server、MySQL 等,但在用户本地运行一个软件的基本需求就是依赖尽量小,本软件仅依赖于 .NET 4.5(是 Windows 自带或必备的运行库),可以说基本上做到了这一点。
