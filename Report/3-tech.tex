% !TeX root = ./report.tex

随着计算机技术的发展,GUI 技术日渐成熟。从最初的表单和控件模式到现在的 MVVM,人们在开发 GUI 的道路上钻研出许多理论完备可以复用的技术。GUI 作为一种面向大部分用户的操作接口已经被广泛认同和使用。并且随着网络技术的稳定发展,大量的图形程序将采用 WebApp、Electron 等形式进行快速开发,基于网页的应用开发逐渐成为软件开发的主流。但是本次毕业设计不使用这种形式,而是用 C\# 编写传统的桌面程序。

本文主要阐述了基于 C\# 和 WPF 的 AGV 地图编辑器的设计和实现。总的来说只有如下几个技术:XAML(用来写界面)、C\#(用来写功能)、LiteDB(用来存储地图数据)、Visual Studio(开发工具)。

\section{C\# 编程语言}

C\# 是一种类型安全的面向对象的编译型脚本语言,开发者可以使用 C\# 开发在 .NET 平台(包括 Windows 和全平台)上运行的安全可靠的应用程序。据最近的新闻,.NET 跨全平台已经在计划之内,预计 2020 年就可以看到,因此使用 C\# 也算是一种紧跟时代的做法。C\# 可以用来编写 Windows 传统桌面应用程序、网站服务(通常是 Asp.NET),以及数据库等。

C\# 的语法的表达能力强并且接近人的思维惯性,因此学习起来不困难。任何熟悉 C、C++ 或 Java 等命令式编程语言的人都可以立刻上手 C\# 的语法。通常情况下,了解上述任何一种语言的开发者可以在很短的时间内就开始使用 C\# 进行高效的工作,而且不需要对 C\# 做任何深入了解。C\# 的语法相对 C++ 来说更加简单、更加强大。举例来说,有 \texttt{nullable}、\texttt{enum}、\texttt{delegate}、\texttt{lambda}(这个在 Java 8 里实现了)等。\texttt{Enumerator} 等集合类接口更是催生出 \texttt{LINQ} 这样简洁而又高效的语言特性和工具。

除了继承、封装、多态这些面向对象的基本特性,使用 C\# 还可以通过以下几个语言特性使开发软件或组件更加轻松:

\begin{itemize}
  \item 通过封装方法签名(即``委托 \texttt{delegate}''),可实现类型安全的事件绑定。
  \item 私有成员变量的修饰符。
  \item 在运行时提供有关类型的声明性元数据的特性,也就是可以支持元编程。
  \item 内联的 XML 文档注释,这在 Java 中也存在。
\end{itemize}

FFI 外部函数接口是一个通用编程语言常用的功能之一,C\# 也不例外。在 Windows 平台上,C\# 可以通过互操作(\texttt{Interop})来与 \texttt{.dll}(在 Linux 平台上它是 \texttt{.so})相互通讯和调用。C\# 的强大之处在于,它甚至可以操作指针以及访问任意内存。

\section{Windows Presentation Foundation}

WPF 于 Windows Vista 时期被提出与采用,它是利用 XAML(一种 XML 语言的扩展版本)、.NET 平台和 MVVM 架构技术实现的图形界面软件开发框架。技术和思想上的先进决定了它的学习曲线相对陡峭,这尤其体现在编写 XAML 时,需要把各种直观的命令式代码翻译到抽象的绑定参数上。除此以外,将界面和逻辑分离也是遵循了 MVC 的设计思想,降低了耦合度。在新版 Visual Studio 中附带了一个 Blend for Visual Studio 软件,就是用来专门设计界面的。

话虽如此,由于受到当时的机器性能以及浏览器原生支持 Flash 等环境影响,WPF 最终没有被广泛地应用在工业领域。这也导致了 WPF 的学习资料寥寥无几,中文资料更是只能找到一两本。

后来随着硬件水平的提高,微软逐渐又提高了对 WPF 的关注并将其使用在自家的多个软件上。再后来 Windows10 出世,微软注意到多平台的发展趋势,提出了 UWP(可以看作 WPF 的跨平台升级版)。

WPF 一路命途多舛,最终存活了下来,相信随着 .NET 跨平台的逐步实现,WPF 还会再迎来一次各界学习的高潮。

\section{Visual Studio}

Visual Studio 作为一款老牌 IDE,已经走到了``宇宙最强''的地位,其强大自不必多言,我这里简单提几个用它开发 WPF 软件时的特点。

在以往开发 Win32 应用程序时,绝大多数的工作都由开发人员来处理,美术设计人员可以涉入的相当有限,而且在 Win32 API 时代,要在用户界面中实现特效的话,都需要撰写大量的代码才可以做到。

在 WPF 中,因为大多数的用户界面元素都可以利用 XAML 来声明,因此微软试着将设计师和开发者结合在一起,以便可以让设计师充份发挥才能,并且开发者也不用在外观设计上伤脑筋,微软的软件设计工具是 Microsoft Expression 系列,其中的 Expression Blend 就是用来交给专业设计师设计 WPF 应用程序界面的工具,现在它叫 Visual Studio Blend。Visual Studio 中也有一个简易的设计工具供开发者使用。

因此 Visual Studio 是开发 WPF 软件的最佳工具。

\section{本章总结}

本章主要介绍了本毕业设计所用的一些核心技术以及一些个人的理解,将开发用到的技术逐一分析优缺点和解释选择它们的原因,学习如何在众多开发技术中选择最适合的来应用到本次毕业设计中。
