% !TeX root = ./report.tex

随着计算机技术的发展,GUI 技术日渐成熟。从最初的表单和控件模式到现在的 MVVM,人们在开发 GUI 的道路上钻研出许多理论完备可以复用的技术。GUI 作为一种面向大部分用户的操作接口已经被广泛认同和使用。并且随着网络技术的稳定发展,大量的图形程序将采用 WebApp、Electron 等形式进行快速开发,基于网页的应用开发逐渐成为软件开发的主流。但是本次毕业设计不使用这种形式,而是用 C\# 编写传统的桌面程序。

本文主要阐述了基于 C\# 和 WPF 的 AGV 地图编辑器的设计和实现。总的来说只有如下几个技术:XAML(用来写界面)、C\#(用来写功能)、LiteDB(用来存储地图数据)、Visual Studio(开发工具)。

\section{C\# 编程语言}

C\# 是一种类型安全的面向对象的编译型脚本语言,开发者可以使用 C\# 开发在 .NET Framework 或 .NET Core 平台上运行的安全可靠的应用程序。C\# 可用于创建 Windows 传统桌面应用程序、XML Web 服务、分布式组件、服务器应用程序、数据库应用程序等。Visual Studio 作为 IDE 提供了代码编辑器、用户界面设计工具、集成调试工具和其他许多工具,以便可以更轻松地开发基于 C\# 语言和 .NET Framework 的应用程序。

C\# 的语法的表达能力极强,但学习起来也很并非困难。任何熟悉 C、C++ 或 Java 等命令式编程语言的人都可以立刻上手 C\# 的语法。通常情况下,了解上述任何一种语言的开发者可以在很短的时间内就开始使用 C\# 进行高效的工作,而且不需要对 C\# 做任何深入了解。C\# 语法简化了 C++ 的许多复杂操作,并提供了许多强大的功能,如值类型可以为 \texttt{null}、枚举 \texttt{enum}、委托 \texttt{delegate}、\texttt{lambda} 表达式和直接访问内存,而 Java 并不提供这些功能。C\# 不仅支持泛型方法和类型,提高了类型安全性和性能,还支持迭代器 \texttt{Enumerator},以便集合类的实现者可以定义方便客户端代码使用的自定义迭代行为。语言集成查询 (\texttt{LINQ}) 表达式构成了最高级的强类型语言查询工具。

C\# 作为面向对象的语言,支持封装、继承和多态这些面向对象应该具有的特性。所有变量和方法(包括作为应用程序入口点的 \texttt{Main} 方法)都封装在类定义中。虽然类只能直接继承一个父类,但可以实现任意数量的接口。若要重写父类中的虚方法,通常必须使用 \texttt{override} 关键字,以免发生意外的重定义,这无疑降低了编程者在 C\# 语言中出错的可能性。在 C\# 中,结构体 \texttt{struct} 就像是一种更轻的类,可以实现接口,但不支持继承。

除了这些面向对象的基本特性,使用 C\# 还可以通过以下几个语言特性使开发软件或组件更加轻松:

\begin{itemize}
  \item 封装方法签名(称作``委托 \texttt{delegate}''),可实现类型安全的事件绑定。
  \item 私有成员变量的修饰符。
  \item 在运行时提供有关类型的声明性元数据的特性,也就是可以支持元编程。
  \item 内联的 XML 文档注释,这在 Java 中也存在。
  \item 语言集成查询 (\texttt{LINQ}),提供各种跨数据源的高级查询功能。
\end{itemize}

如果需要与其他 Windows 软件(如 COM 对象或本机 Win32 DLL)进行交互,可以在 C\# 中通过名为``互操作 \texttt{Interop}''的过程来实现。借助互操作,C\# 程序可以执行本机 C++ 应用程序可以执行的几乎任何操作。在可能需要对内存进行直接访问的情况下,C\# 甚至支持指针和``不安全 \texttt{unsafe}''的代码。

C\# 编译过程比 C 和 C++ 更简单,比 Java 更灵活。没有单独的头文件,也不要求按特定顺序声明方法和类型。C\# 源文件可以定义任意数量的类、结构体、接口和事件。

因此使用 C\# 这种强大的编程语言来开发本软件简直是杀鸡用牛刀。

\section{.NET Framework 平台体系结构}

C\# 程序在 .NET Framework 上运行,这是 Windows 不可或缺的一部分,包括名为``公共语言运行时 (CLR)''的虚执行系统和一组统一的类库。CLR 是 Microsoft 研发的公共语言基础结构 (CLI) 的商业实现,CLI 是作为执行和开发环境(语言和库在其中无缝对接)创建依据的国际标准。

用 C\# 编写的源代码被编译成符合 CLI 规范的中间语言 IL。IL 代码和资源(如位图和字符串)存储在磁盘上名为``程序集''的可执行文件(扩展名通常为 \texttt{.exe} 或 \texttt{.dll})中。程序集包含一个介绍程序集的类型、版本、区域和安全要求的清单。

当 C\# 程序执行时,程序集会加载到 CLR 中,根据清单中的信息执行各种操作。然后,如果满足安全要求,CLR 会直接执行实时 (JIT) 编译,将 IL 代码转换成本机指令。CLR 还提供其他与自动垃圾回收、异常处理和资源管理相关的服务。CLR 执行的代码有时称为``托管代码'',被编译成面向特定系统的本机语言。

语言互操作性是 .NET Framework 的一项重要功能。由于 C\# 编译器生成的 IL 代码符合公共类型规范 (CTS),因此 C\# 生成的 IL 代码可以与 .NET 版本 Visual Basic、Visual C++ 或其他任何符合 CTS 的超过 20 种语言生成的代码进行交互。一个程序集可能包含多个用不同 .NET 语言编写的模块,且类型可以相互引用,就像是用同一种语言编写的一样。

除了运行时服务之外,.NET Framework 还包括一个由 4000 多个已整理到命名空间中的类构成的扩展库,这些类提供各种实用功能,包括文件输入输出、字符串控制、XML 分析和 Windows 窗体控件。典型的 C\# 应用程序广泛使用 .NET Framework 类库来处理常见的``管道''等零碎工作。

因此使用 .Net 平台就可以获得大量原生标准库的支持,配合 C\# 进行开发就是如虎添翼。

\section{Windows Presentation Foundation}

WPF 是微软推出的 .NET Framework 3.0 及以后版本语言平台的组成部分之一,它是一套基于 XAML、.NET Framework、向量绘图技术的展示层开发框架,微软视其为下一代用户界面技术,广泛被用于 Windows Vista 的界面开发。

WPF 使用一种名为 XAML(eXtensible Application Markup Language)的标记语言来开发界面,这使得界面开发和后台逻辑能被很好的分离,降低了耦合度,使用户界面设计师与程序开发者能更好的合作,降低了维护和更新的成本。

WPF/E 是 WPF 的子集合,全名是:Windows Presentation Foundation Everywhere。在 XAML 与 JavaScript 充分结合之下,可跨越各种平台,后来 WPF/E 演化为 Microsoft Silverlight,担负微软在丰富互联网应用程序领域,并正面与 Adobe Flash 竞争的产品。然而因为用户太少,开发者寥寥无几,Silverlight 已经先于 Flash 夭亡了。Flash 后来也在 HTML5 的发展下从网页前端离开了。如果说现在应该用什么来开发网页前端领域的富客户端应用的话,那必然是 HTML5 了,本次毕业设计不涉及网页前端等领域,因此可以采用 WPF 来实现。

因此 WPF 用来设计 GUI 可以获得很好的界面效果,用来开发本毕业设计十分合适。

\section{Visual Studio}

在以往开发 Win32 应用程序时,绝大多数的工作都由开发人员来处理,美术设计人员可以涉入的相当有限,而且在 Win32 API 时代,要在用户界面中实现特效的话,都需要撰写大量的代码才可以做到。

在 WPF 中,因为大多数的用户界面元素都可以利用 XAML 来声明,因此微软试着将设计师和开发者结合在一起,以便可以让设计师充份发挥才能,并且开发者也不用在外观设计上伤脑筋,微软的软件设计工具是 Microsoft Expression 系列,其中的 Expression Blend 就是用来交给专业设计师设计 WPF 应用程序界面的工具,现在它叫 Visual Studio Blend。Visual Studio 中也有一个简易的设计工具供开发者使用。

因此 Visual Studio 是开发 WPF 软件的最佳工具。

\section{本章总结}

本章主要介绍了本毕业设计所用的一些核心技术以及一些个人的理解,将开发用到的技术逐一分析优缺点和解释选择它们的原因,学习如何在众多开发技术中选择最适合的来应用到本次毕业设计中。
