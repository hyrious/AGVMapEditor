% !TeX root = ./report.tex

GUI(图形用户接口)技术从最早的 WCF、WinForm 到现在原生跨平台的 Qt、Electron 一直是一个经久不衰的话题。举例来说,Google 最近正在大力推广他的新技术 Flutter,可以快速在 iOS 和 Android 上构建高质量的原生用户界面。究其风靡工业界学术界的原因,是因为一个用户体验良好的 GUI 应用本身十分复杂,各领域的开发者都想要创造某种范式\cite{url:uiArchs}来进行简化。然而不管如何创造新控件、创造新 DSL(领域特定语言),每个用例的逻辑细节都是需要人来手工输入的,这就导致了其代码复杂度基本不可能降低。

WPF、Qt 等框架的出现和热门的原因十分清晰:不同于花大力气直接查阅调用系统 API(参考 WCF 开发),这些封装好的 GUI 技术可以大大缩短开发/调试时间,而且有的还能做到跨平台。

然而,世上没有绝对完美的技术,GUI 开发也是如此:Qt 的 slot 模式十分方便,使用 C++ 也十分高效,但是一个 Hello world 都要带上几十 M 的 qt5.dll,开发难度完全由开发者写 C++ 的熟练度决定。WPF 的 MVVM 模式虽然自称优秀,但是要在其架构下编写良构的代码并不容易,更别提他还推荐把代码逻辑都写成 Binding;另外由于他是基于 .Net 平台,内存分配基本上就是个笑话,任何一次频繁的字符串操作就要瞬间吃掉大量内存。Electron 等利用网页开发的技术可以说是相当便宜了(指学习难度),然而每次启动程序都是启动一次浏览器的成本属实高昂。

本章会介绍开发基于 WPF 的地图编辑器的背景及其意义,然后介绍本软件的主要功能以及使用方法,最后介绍 WPF 开发存在的一些问题。

\section{开发背景}

随着工业技术的发展,各类机器人渐渐代替了大量手工作业。我国 AGV 机器人在各项政策的保驾护航下,迅速占领着大量市场份额。《2018 -- 2023年中国仓储物流行业市场前瞻与投资战略规划分析报告》数据\cite{url:20180131C02LR0}显示,我国对 AGV 机器人的需求增长迅速,基本上集中在应用领域。

\begin{figure}[H]
  \centering
  \includegraphics[width=0.618\textwidth]{assets/bar.pdf}
  \caption{2013 -- 2018 年中国 AGV 机器人销量(单位:台)}
  \label{fig:figure1}
\end{figure}

计算机技术的飞速发展,带动了一系列传统行业的转型升级。例如制造业常用的叉车,就会被 AGV 甚至 AMR(自主移动机器人)替代。

\section{国内外研究现状}

\section{主要功能概述}

\section{本次研究的意义}
